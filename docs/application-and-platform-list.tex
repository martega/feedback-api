%%%%%%%%%%%%%%%%%%%%%%%%%%%%%%%%%%%%%%%%%%%%%%%%%%%%%%%%%%%%%%%%%%%%%%%%%%%%
%                    application-and-platform-list.tex                     %
%%%%%%%%%%%%%%%%%%%%%%%%%%%%%%%%%%%%%%%%%%%%%%%%%%%%%%%%%%%%%%%%%%%%%%%%%%%%

\section{API Calls}
\begin{center}
\begin{tabular}{|l||l||l|}
\hline

\multicolumn{1}{|c||}{\textbf{API Call}} &
\multicolumn{1}{c||}{\textbf{HTTP Method}} &
\multicolumn{1}{c|}{\textbf{URI}} \\

\hline
\hline
get application and platform list  & GET & /applications \\
\hline
\end{tabular}
\end{center}

\section{Get Application and Platform List}

Returns a list of all tracked applications and their respective platforms.
An application and platform are assumed to be in the system if there exists
at least one user for that application and platform.

\subsection{Request Format}
\begin{center}
\begin{tabular}{|l||l|}
\hline
HTTP Method & GET           \\
\hline
URI         & /applications \\
\hline
Query Parameters & N/A           \\
\hline
Body        & N/A           \\
\hline
\end{tabular}
\end{center}


\subsection{Response Format}

The response body for this API call is a JSON object with a single field, \textbf{applications},
which is a list of application objects. Every application object has two fields, \textbf{name},
and \textbf{platforms}. The \textbf{name} field is the name of the appliction, and the
\textbf{platforms} field is a list of all platforms that the application has users for.

\subsection{Examples}

\textbf{GET} http://feedback-api.cloudapp.net/applications
\begin{verbatim}
{
    "applications":[
        {
            "name":"employee_app",
            "platforms":[
                "ios"
            ]
        },
        {
            "name":"ichiba",
            "platforms":[
                "win8",
                "web",
                "ios",
                "android"
            ]
        }
    ]
}
\end{verbatim}
